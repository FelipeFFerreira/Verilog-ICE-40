\documentclass{article}
\usepackage[utf8]{inputenc}
\usepackage{lmodern}
\usepackage[T1]{fontenc}
\usepackage{minted}
\usepackage[margin=1in]{geometry}
\usepackage{graphicx}
\usepackage{wrapfig}
\usepackage{multicol}
\setlength{\columnsep}{-0.5cm}

\begin{document}

\input{headerLab}
%%Versao Lab
    \begin{flushleft}
       {\large \textbf {\\[0.02in] Laboratório 1 - Preparação de ambiente}}
       { \textsc {\newline Versão 1 - XX/XXXX}}
    \end{flushleft}

%%Atividades Lab
\section{Objetivos}
 
\subsection{Objetivos específicos}
\begin{itemize}
    \item Preparar o ambiente para execução das atividades em laboratório;
    \item Testar configurações realizadas a partir de um programa "blink" ou "piscar".
    
\end{itemize}
\section{Pré-requisitos}
\begin{itemize}
    \item Sistema Operacional Linux Ubuntu (mínimo: versão 20.4);
    \item Placa ICEsugar-nano com conector USB;
    \item Drivers FTDI.
    \textnormal \newline{\vspace{-0.5mm}}
    \textnormal \newline{Caso não tenha os \textit{drivers} instalados, use os seguintes comandos no terminal:}
    
    \begin{minted}{bash}
        mkdir ftdi
        cd ftdi
        wget https://www.ftdichip.com/Drivers/D2XX/Linux/libftd2xx-x86_64-1.4.8.gz
        tar xfvz libftd2xx-x86_64-1.4.8.gz
        cd release/build/
        sudo -s 
        cp libftd2xx.* /usr/local/lib
        chmod 0755 /usr/local/lib/libftd2xx.so.1.4.8
        ln -sf /usr/local/lib/libftd2xx.so.1.4.8 /usr/local/lib/libftd2xx.so
        exit
    \end{minted}
    
\end{itemize}
\section{Atividades}



\subsection{Instalação de \textit{Drivers} e demais pacotes}
\textnormal{Primeiro será realizada a instalação dos pacotes para Ubuntu, abra o terminal na pasta pessoal ou em outro destino que preferir, e use os seguintes comandos:}

\begin{minted}{bash}
    sudo apt-get install build-essential clang bison flex libreadline-dev \
                    gawk tcl-dev libffi-dev git mercurial graphviz   \
                    xdot pkg-config python python3 libftdi-dev \
                    qt5-default python3-dev libboost-all-dev cmake libeigen3-dev
\end{minted}


\subsection{Instalação das ferramentas}
\begin{itemize}

    \item \textbf{Icestorm Toolchain}
    \textnormal{\newline O conjunto de ferramentas icestorm, empregue de acordo com o FPGA da placa ICESugar-nano, pode ser instalado com os comandos:}

    \begin{minted}{bash}
        git clone https://github.com/YosysHQ/icestorm.git icestorm
        cd icestorm
        make -j$(nproc)
        sudo make install
    \end{minted}
    
    \item Retorne a pasta inicial por meio do comando:
    \begin{minted}{bash}
       cd ..
    \end{minted}
    \vspace{08pt}



    \item \textbf{NextPNR}
    \textnormal{\newline O NextPNR é utilizado como ferramenta para roteamento e posicionamento. Utilize os seguintes comandos:}

    \begin{minted}{bash}
      git clone https://github.com/YosysHQ/nextpnr nextpnr
      cd nextpnr
      cmake -DARCH=ice40 -DCMAKE_INSTALL_PREFIX=/usr/local .
      make -j$(nproc)
      sudo make install
    \end{minted}

    \textnormal{Aguarde a instalação, poderá levar alguns minutos. Caso necessário o emprego da GUI (interface gráfica), adicione a seguinte informação na linha cmake:}
    \begin{minted}{bash}
       -DBUILD_GUI=ON
    \end{minted}
    \vspace{-0.5mm}
    \textnormal{ Obs.: Sempre que alterar as ferramentas Icestorm, também efetue um \textit{rebuild} do NextPNR.}
    \item Retorne a pasta inicial da instalação.
    \vspace{08pt}
    
    
    
    \item {\textbf{Yosys}}
    \textnormal\newline{O Yosys será aplicado na síntese do código em Verilog, para instalação use:}
    \begin{minted}{bash}
        git clone https://github.com/YosysHQ/yosys.git yosys
        cd yosys
        make -j$(nproc)
        sudo make install
        \end{minted}
        
        \item Lembre-se de voltar ao destino original.
        \vspace{08pt}
        
    \item {\textbf{Icarus Verilog + Gtkwave}}
    \textnormal\newline{O Icarus mais o software Gtkwave irão servir como simulador, para instalação use:}
    \begin{minted}{bash}
        sudo apt install iverilog gitkwave
        \end{minted}
        
        \item Lembre-se de voltar ao destino original.
        \vspace{08pt}
        
\end{itemize}


\section{Teste de ferramentas}
\textnormal{Para verificar se o conjunto instalado está funcionando adequadamente, vamos realizar um teste com o LED laranja da placa. O código originalmente escrito por wuxx$^{[1]}$ foi adaptado para uso em lab:}
\begin{itemize}
    \item Crie uma pasta localizada dentro do destino escolhido para instalação, use esta pasta para salvar os próximos arquivos solicitados;
    \item Implemente o código seguinte em um arquivo chamado "piscar" de extensão verilog .v:
    
    \begin{minted}{verilog}
        module switch(   input CLK,
                         output LED
                     );
      
            reg [25:0] counter;

            assign LED = ~counter[21];

            initial begin
                counter = 0;
            end

            always @(posedge CLK)
            begin
                counter <= counter + 1;
            end

        endmodule
    \end{minted}
    
    
    \item Implemente o código a seguir em um arquivo chamado "io" de extensão .pcf:
    \begin{minted}{bash}
        #   BOARD PINS

        set_io LED B6

        set_io CLK D1

        #   PMOD FULL (PMOD3)
        set_io RX  A3
        set_io TX  B3
        set_io --warn-no-port PMOD1 B4
        set_io --warn-no-port PMOD2 C6
        set_io --warn-no-port PMOD3 B5
        set_io --warn-no-port PMOD4 E3
        set_io --warn-no-port PMOD5 E1
        set_io --warn-no-port PMOD6 C2
        set_io --warn-no-port PMOD7 B1
        set_io --warn-no-port PMOD8 A1
        
        #   PMOD LEFT (PMOD2)
        set_io --warn-no-port PMODL1 B3
        set_io --warn-no-port PMODL2 A3
        set_io --warn-no-port PMODL3 B6
        set_io --warn-no-port PMODL4 C5

        #   PMOD RIGHT (PMOD1)
        set_io --warn-no-port PMODR1 A1
        set_io --warn-no-port PMODR2 B1
        set_io --warn-no-port PMODR3 D1
        set_io --warn-no-port PMODR4 E2
        
    \end{minted}
    
    \item Crie um arquivo \textit{makefile} com os seguintes comandos:
    
    \begin{minted}{bash}
    filename = piscar
    pcf_file = io.pcf

    ICELINK_DIR=$(shell df | grep iCELink | awk '{print $$6}')
    ${warning iCELink path: $(ICELINK_DIR)}

    build:
    	yosys -p "synth_ice40 -json $(filename).json -blif $(filename).blif" $(filename).v
    	nextpnr-ice40 --lp1k --package cm36 --json $(filename).json --pcf $(pcf_file) --asc $(filename).asc --freq 48
    	icepack $(filename).asc $(filename).bin

    prog_flash:
	    @if [ -d '$(ICELINK_DIR)' ]; \
            then \
                cp $(filename).bin $(ICELINK_DIR); \
            else \
                echo "iCELink not found"; \
                exit 1; \
        fi


    clean:
	    rm -rf $(filename).blif $(filename).asc $(filename).bin
    \end{minted}
    
    \item Abra a pasta dentro do terminal de comando e execute o seguinte comando:    
    \begin{minted}{bash}
        make build
    \end{minted}
    \item Verifique os resultados, caso nenhum erro seja identificado, utilize:
     \begin{minted}{bash}
        make prog_flash
    \end{minted}
    \item Com isso, a LED laranja deve piscar em uma determinada frequência.
\end{itemize}

\section{Exercícios sugeridos}
\begin{enumerate}
    \item Abra o arquivo piscar.v e na linha 8, e substitua "\textasciitilde counter[23]" por "\textasciitilde counter[19]", o que ocorre com a LED?
    \item O arquivo "io.pcf" contém algumas informações relevantes utilizadas pelas ferramentas, sabe dizer o que "io.pcf" descreve?
    
\end{enumerate}

\section*{Referencias}
\textnormal{\footnotesize{[1] https://github.com/wuxx/icesugar-nano}}
\end{document}
